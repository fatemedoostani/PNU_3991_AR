\documentclass{book} 
\usepackage[top=3cm,right=3.5cm,bottom=3cm,left=3cm]{geometry}
\renewcommand{\baselinestretch}{1.7}
\parindent=5pt
\begin{document}
page $58$\\\\
without asking the users' permission that allowed the researcher (Marty Rimm) to track approximately four thousand individuals who accessed erotic and/or nonerotic Usenet groups once a month or more.\\
Among the many ethical issues illustrated in this case (such as gathering data from privileged sources for which there was no evidence that permission was attained), is the continued debate over what is a public forum and what is a private forum. VVhile there is general consensus that Usenet groups are public forums and anyone can read them, it is less clear if studying those who subscribe to the newsgroups can be done without informed consent or loss of personal integrity, dignity, and respect for privacy. Are online interactions on bulletin boards and Usenet groups public, or are they private conversations that occur in public spaces? And if the conversations are private, but occurring in a public space, are researchers intruding? And if we do agree that researchers are intruding, what is the nature of the intrusion as it relates to harm that may or may not result from the research and in light of the benefits that may be realized from the findings of the study?\\
As the Rimm study illustrates, the Net has become the focus of an extensive range of ethical concerns as it evolves into a modern agora in which contrasting ethical interpretations and actions are played out. This context is further complicated by characteristics of the Net that are unique and largely unknowable within physical environments. These include the capacity of individuals to retain various degrees of anonymity, the capacity to horde, steal, and retain objecß with no physical embodiment, the blurring between public and private domains, the possibility of participating in sexual experiences with no physical contact, and other forms of interpersonal activicy that have few parallels in the physical world. These evolving scenarios and the actors (both physical and virtual) create the fascinating context in which the e-researcher works. To date, there are no clearly defined criteria for appropriate ethical behavior for all e-researchers or all e-research activity For many behaviors and challenges we experience on the Web, we search for successful and appropriate rules from the pre-Net context to use as analogies and models that we can applv to define appropriate ethics for the Net. Existing rules provide a good starting point, but they are not sufficient to capture all of the peculiar ethical nuances and concerns of Netbased research activities and subsequent research.\\
One way to overcome many of these issues is to develop a hill set of ethical guidelines and practices specifically designed for the Net. However, because the field and methodology of study, as well as the actual context of activity are changing, there is need for caution before formalizing permanent ethical standards that may not meet the longterm needs of either researchers or participants. Currently, most efforts to prescribe ethical behavior in Netbased contexts attempt to apply existing ethical codes of practice. We explore both the value and limitations of this type of proiective ethics throughout this chapter.\\
\textbf{Different Applications Present Different Challenges} \\
 As described in Chapter 1 (Figure 1.1), there are two quite distinct ways to use the Net for research. Both ways present ethical challenges for the e-researcher.\newpage
 page$59$\\\\
 In the case in which e-research methods are used to study behavior that does not take place on the network. but the network is used as a means to contact, observe, survey, or interview participants, the ethical issues are more directly related, and analogous, to non-networked research. In this case, ethical guidelines can, generally, be ported from existing ethical guidelines. Some of the procedures involved in obtaining and verifying consent and providing for security and privacy are unique to the network, but, generally, updating existing guidelines to a networked context is relatively straightforward. However, when applying e-research to study behavior that takes place on the networks, ethical issues become more complex. In these virtual contexts, issues of identity, anonymity, privacy, and protection of virtual self are difficult to transfer to the online environment, and direct analogies from non-networked research ethics become stretched and in some cases nonfunctional. The following sections discuss these difficulties as they relate to existing ethical guidelines and principles.\\
\textbf{Standing Ethical Guidelines and Principles} \\
In addition to dignity, privacy, self-esteem, and our own values, ethics revolves around the behavior of the individual in regard to that demanded or accepted by a larger group (Thomas, 1996; Thomas, 1999). These dictates from the larger group are often proscribed as rules of conduct for mem bers of particular professions or organizations. Traditionally. the principles for such "right" conduct in most societies are derived from sacred texts or oral teachings. However, Western societies are characterized by diversity of culture, where there is no one universal teaching to provide us with these guidelines. Societies and groups wishing to act with particular authority are forced to look to sets of commonlv held and defined values to determine what is correct behavior in both the general and the individual case.\\
As mentioned, ethical behavior stems from commonly perceived and held values. With respect to research, the commonly perceived and held values generally relate to the following three areas: autonomy and respect for other persons, the reduction of potential to harm, and, conversely, the potential for beneficence and justice. Researchers and their professional bodies, such as the American Psychological Association, or hinding bodies, such as the Canadian Government Granting Councils, have built on these values to create sets of ethical guidelines for activity in the physical realm. These are detailed by organizations such as the American Psychology Association (APA) at http://uww.apa.org/ethics/code.html. H'hile proscribed for practicing psychologists, many researchers from other disciplines (e.g., education and the social sciences) usually adhere to these general principles as well. We outline these principles here, adjusting them toward a generalized perspective for e-researchers.\\
\textbf{Competence.}  E-researchers should strive to maintain high standards of competence in their work. They need to recognize the boundaries of their particular competencies and the limitations of their expertise and provide only those services and use only those techniques for which they are qualified by education, training, or experience. In areas in which recognized professional standards do not yet exist, researchers should exercise careful judgment and take appropriate precautions to protect the welfare of those with whom they work. \newpage
page $60$\\\\
Moreover, they should maintain knowledge of relevant scientific and professional information, and recognize the need for ongoing education.\\
\textbf{Integrity.}  As e-researchers seek to promote integrity in their research, they are honest, fair, and respectful of others. In describing or reporting their qualifications, they do not make statements that are false, misleading. or deceptive. E-researchers should strive to be aware of their own belief systems, values, needs, and limitations and the effect of these on their work. To the extent feasible, they attempt to clarify for relevant parties the roles they are performing and to function appropriately in accordance with those roles and avoid improper and potentially harmful dual relationships.\\
\textbf{Professional and Scientific Responsibility.}  E-researchers should uphold professional standards of conduct, clarify their professional roles and obligations, accept appropriate responsibility for their behavior. and adapt their methods to the needs of different populations. Further, they should consult with, refer to, and/or cooperate with other professionals and institutions to the extent needed to serve the best interests of their research participants. Just as they are for any other person. moral standards and conduct are personal matters for the e-researcher, except when their actions compromise their professional responsibilities or reduce the public's trust. appropriate, they consult with colleagues to prevent or avoid unethical conduct.\\
\textbf{Respect for People's Rights and Dignity.}  E-researchers hold appropriate respect to the fundamental rights, dignity, and worth of all people. They respect the rights of individuals to privacy, confidentiality, self-determination, and autonomy, mindful that legal and other obligations may lead to inconsistency and conflict with the exercise of these rights. Moreover, they are aware of cultural, individual, and role differences, including those due to age, gender, race, ethnicity, national origin, religion, sexual orientation, disability, language, and socioeconomic status. They try to eliminate the effect of biases based on those factors on their work, and they do not knowingly participate in or condone unfair discriminatory practices.\\
\textbf{Concern for Others' Welfare.} E-researchers seek to contribute to the welfare of those with whom they interact professionally. In their professional actions, they weigh the welfare and rights of their research participants, and other affected persons. IVhen conflicts occur among colleagues' obligations or concerns, e-researchers attempt to resolve these conflicts and to perform their roles in a responsible fashion that avoids or minimizes harm, they are sensitive to real and ascribed differences in power between themselves and others, and they do not exploit or mislead other people during or after professional relationships.\\
\textbf{Social Responsibility.}  E-researchers are aware of their professional and scientific responsibilities to the community and the society in which they work and live. They apply and make public their knowledge of their profession or field of study to contribute to human welfare. They are concerned about and work to mitigate the causes of human suffering. E-researchers comply with the law and encourage the development..
\end{document}